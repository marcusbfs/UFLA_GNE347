\chapter{Descrição do Projeto}\label{descricao}

\section{Materiais utilizados}

% Please add the following required packages to your document preamble:
% \usepackage{booktabs}
\begin{table}[H]
\centering
\caption{Materiais requeridos para o projeto}
\label{tabela:custo}
\begin{tabular}{@{}lr@{}}
\toprule
\multicolumn{1}{c}{\textbf{Materiais}}  & \multicolumn{1}{c}{\textbf{Custo (R\$)}} \\ \midrule
Tubos PVC ($\SI{50}{\mm}$)              & 7,00/m                                   \\
Joelho $\ang{90}$ ($\SI{50}{\mm}$)      & 3,00/unidade                             \\
Tubos de ensaio                         & 5,00/unidade                             \\
Garras                                  & 25,00/unidade                            \\
Suportes universais                     & 50,00/unidade                            \\
Papel milimetrado                       & 0,50/folha                               \\
Tela de tecido                          & 4,00/m                                   \\
Termômetro                              & 130,00/unidade                           \\
Relógio                                 & -                                        \\
Sílica gel                              & 29,00/kg                                 \\
Água                                    & -                                        \\
Acetona                                 & 40,00/L                                  \\
Etanol                                  & 22,00/L                                  \\ \bottomrule
\end{tabular}
\end{table}


\section{Procedimento experimental}

Inicialmente o tubo de PVC deverá ser seccionado, por uma furadeira, no formato
circular dos tubos de ensaio que serão usados para armazenar as soluções
analisadas.

O tubo então deverá ser fixado em hastes metálicas, e nas entradas de ar do tubo
deverão ser instalados telas contendo sílica, a fim de evitar a passagem de
umidade externa pelo tubo.

Pretende-se realizar testes com pelo menos três tipos de substâncias dispostas
simultaneamente em tubos de ensaios com escala milimetrada. Serão testadas a
influência do tipo de convecção natural e forçada (com o auxílio de um secador
de cabelos), e a influência da temperatura (os tubos de ensaio deverão estar
submersos em banho maria a temperatura constante).


%%% Local Variables:
%%% mode: latex
%%% TeX-master: "../main_archive"
%%% End:
